
% Thesis Abstract -----------------------------------------------------

% ok
%\begin{abstractslong}    %uncommenting this line, gives a different abstract heading
\begin{abstracts}        %this creates the heading for the abstract page

% Generic object recognition
In its essence, machine perception aims to extract interpretable structured infortmation from unstructured signals.
Object recognition is a foundational task for computer vision and machine perception more broadly.
Since the remarkable success of AlexNet in the ILSVRC2012 competition, 
Convolutional Neural Networks (CNN) have allowed for unprecedent progress in object recognition,
which has opened the door for new applications that were previously though impossible. 
CNN-based classifiers have become the backbone of modern computer vision.
Complex vision systems from object detection and image segmentation systems to 
higher level models such as image captioning and Visual Question Answering systems, 
have all been built on top of the backbone architecture of CNN classifiers.

% CNN limitations
Given this success and the central place of CNN-based object recognition components in vision systems, 
it is important to think about their limitations.
On the conceptual side, object recognition is currently framed as a supervised classification problem.
This classification setting induces a closed world assumption: 
The set of object categories a model can recognize is finite and fixed both by the architecture and the available training data.
Outside 
Data annotation problem.
Closed world

In comparison, humans flexibility.
Open world.
Because we combine perceptual abilities with higher abstraction formalisms and reasoning.
For instance, children can recognize zebra.

% Promises of ZSL
The analogy to the human ability to recognize unknown obects has motivated ZSL.
ZSL do XXX
From a practical perspective, XXX.
From a research point of view, XXX.

% Current state
Despite its great potential impact and after a decade of active research, XXX.
In this paper, we XXX




\end{abstracts}
%\end{abstractlongs}


% ---------------------------------------------------------------------- 
